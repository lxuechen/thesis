\section[Dimension-Independence via Restricted Lipschitz Continuity]{Dimension-Independence via \\Restricted Lipschitz Continuity}\label{sec:theory}
In this section, we introduce the restricted Lipschitz continuity condition and derive improved bounds for the excess empirical and population risks for DP-SGD when optimizing convex objectives.

\begin{defi}[Restricted Lipschitz Continuity]
We say that the loss function $h: \cK \to \R$ is restricted Lipschitz continuous with coefficients $\{G_k\}_{k=0}^d$, if
for all $k \in \{0, \dots, d\}$, there exists an orthogonal projection matrix $P_k$ with rank $k$ such that 
\begin{align*}
    \|(I-P_k)\nabla h(x) \|_2\le G_k,
\end{align*}
for all $x \in \cK$ and all subgradients $\nabla h(x) \in \partial h(x)$.
\end{defi}

Note that any $G$-Lipschitz function is also trivially restricted Lipschitz continuous with coefficients $G = G_0 = G_1 = \cdots = G_d$, since orthogonal projections never increase the $\ell_2$-norm of a vector (generally, it is easy to see that $G = G_{0}\ge G_{1}\geq G_{2}\geq\cdots\geq G_{d}=0$). 
On the other hand, we expect that a function which exhibits little growth in some subspace of dimension $k$ to have a restricted Lipschitz coefficient $G_{d-k}$ of almost 0.
Our bounds on DP convex optimization characterize errors through the use of restricted Lipschitz coefficients. 
We now summarize the main assumption.
\begin{assu}
\label{assm:G_k}
The per-example loss function $f(\cdot; s)$ is convex and $G$-Lipschitz continuous for all $s \in \mathcal{S}$.
The empirical loss $F(\cdot; \cD)$ is restricted Lipschitz continuous with coefficients $\{G_k\}_{k = 0}^d$.
\end{assu}

\subsection{Bounds for Excess Empirical Loss}
We present the main theoretical result on DP-ERM for convex objectives. 
The result consists of two components:
Equation~\eqref{eq:erm_general_condition} is a general bound that is applicable to any sequence of restricted Lipschitz coefficients;
Equation~\eqref{eq:erm_fast_decaying} specializes the previous bound when the sequence of coefficients decays rapidly and demonstrates dimension-independent error scaling.
\begin{theo}[Excess Empirical Loss]{theorem}{DPERM}
\label{thm:DPERM}
Let $ \delta \in (0, \frac{1}{2}]$ and $\epsilon\in(0,10]$.
Under Assumption~\ref{assm:G_k}, for all $k \in [d]$, 
setting 
$T=\Theta(n^2 + d \log^2 d)$, $\sigma=\Theta \bracks{ \frac{\sqrt{T\log(1/\delta)}}{n\epsilon} }$, $\eta=\sqrt{\frac{D^{2}}{T\cdot G_{0}^{2}\cdot k\sigma^{2}}}$
and $\alpha=\frac{1}{D}\sqrt{\sum_{s=1}^S s^2 2^{s}G_{2^{s-1}k}^{2}}$, where $S = \lfloor \log (d / k) \rfloor + 1$, DP-SGD (Algorithm~\ref{alg:noisy_SGD}) is $(\epsilon,\delta)$-DP, and
\begin{align}
\label{eq:erm_general_condition}
\E \sbracks{  F(\overline{x};\cD)-  \min_x F(x;\cD) } &~ \lesssim \frac{G_{0}D\sqrt{k\log(1/\delta)}}{\epsilon n}+D\sqrt{\sum_{s=1}^{S}s^2 2^{s}G_{2^{s-1}k}^{2}},
\end{align}
where $\|x^{(0)}- \argmin_x F(x; \cD) \|_2 \le D$, $\ox$ is the (random) output of DP-SGD (Algorithm~\ref{alg:noisy_SGD}), and the expectation is over the randomness of $\overline{x}$.
Moreover, if for some $c>1/2$, we have $G_k\leq G_0 k^{-c}$ for all $k \in [d]$, and in addition $ n\geq \epsilon^{-1} \sqrt{\log(1/\delta)}$,
minimizing the right hand side of \eqref{eq:erm_general_condition} with respect to $k$ yields
\begin{align}
\label{eq:erm_fast_decaying}
\E \sbracks{  F(\overline{x};\cD)-  \min_x F(x;\cD) } \lesssim G_{0}D \cdot \bracks{ 
        \frac{\sqrt{\log(1/\delta)}}{\epsilon n}
    }^{2c/(1+2c)
}.
\end{align}
\end{theo}
We include a sketch of the proof techniques in Subsection~\ref{sec:proof_technique} and defer the full proof to Subsection~\ref{app:dperm_proof}.
\begin{remark}
Consider DP-ERM for learning generalized linear models with convex and Lipschitz losses.
When the (empirical) data covariance is of rank $r < d$, the span of gradients $\text{span}(\{ \nabla_x F(x) \})$ is also of rank $r$. 
Thus, the average loss is restricted Lipschitz continuous with coefficients where $G_{r'} = 0$ for all $r' > r$. 
Setting $k = r$ in \eqref{eq:erm_general_condition} yields an excess empirical risk bound of order $O\bracks{ {G_{0}D\sqrt{ r \cdot\log(1/\delta)}} \epsilon^{-1} n^{-1} }$.
This recovers the previous dimension-independent result by~\cite{song2021evading}.
\end{remark}

% \begin{remark}
%     If $\epsilon n<\sqrt{\log(1/\delta)}$, instead of running DP-SGD, one can output the public initial point $x_0$ directly and get the excess empirical risk $G_0D$.
% \end{remark}


The restricted Lipschitz continuity condition can be viewed as a generalized notion of rank.
The result captured in \eqref{eq:erm_fast_decaying} suggests that the empirical loss achieved by DP-SGD does not depend on the problem dimension if the sequence of restricted Lipschitz coefficients decays rapidly. 
We leverage these insights to build intuition for understanding privately fine-tuning language models in Section~\ref{sec:experiments}.
% The quick decaying phenomena of gradient (i.e. $G_{k}\leq \frac{G_{0}}{k^{c}}$) seems to be very common in practice.
% Our experiment suggests $c = 1$ for the pretrained language model we used. 


\subsection{Bounds for Excess Population Loss}
%Making use of the stability of SGD, we can bound the generalization error and hence the excess population loss for the DP-SCO problem.
%To prove the utility guarantees in Theorem \ref{thm:DPSCO} 
For DP-SCO, we make use of the \emph{stability} of DP-SGD to bound its generalization error \citep{BE02}, following previous works \citep{bassily2019private,BFGT20,song2021evading}.
The bound on the excess population loss follows from combining the bounds on the excess empirical risk and the generalization error.

\begin{restatable}[Excess Population Loss]{theorem}{DPSCO}
\label{thm:DPSCO}
Let $ \delta \in (0, \frac{1}{2}]$ and $\epsilon\in (0,10]$.
Under Assumption~\ref{assm:G_k}, for all $k \in [d]$, 
by setting
$T=\Theta(n^2 + d \log^2 d)$,
$\sigma=\Theta \bracks{ \frac{\sqrt{T\log(1/\delta)}}{n\epsilon} }$, 
$\eta=\sqrt{\frac{D^{2}}{T\cdot G_{0}^{2}(T/n+k\sigma^{2})}}$
and $\alpha=\frac{1}{D}\sqrt{\sum_{s=1}^S s^2 2^{s}G_{2^{s-1}k}^{2}}$, where $S = \lfloor \log(d/k) \rfloor + 1$, 
DP-SGD (Algorithm~\ref{alg:noisy_SGD}) is $(\epsilon,\delta)$-DP, and
\begin{align*}
\E \sbracks{
    F(\ox; \cP) - \min_{x} F(x; \cP)
}
\lesssim
    \frac{G_0D}{\sqrt{n}}+\frac{G_{0}D\sqrt{k\log(1/\delta)}}{\epsilon n}+D\sqrt{\sum_{s=1}^{S}s^2 2^{s}G_{2^{s-1}k}^{2}},
\end{align*}
where $\|x^{(0)}- \argmin_x F(x ; \cP) \|_2 \le D$, $\ox$ is the (random) output of DP-SGD (Algorithm~\ref{alg:noisy_SGD}), and the expectation is over the randomness of $\overline{x}$.

Moreover, if for some $c>1/2$, we have $G_k\leq G_0 k^{-c}$ for all $k \in [d]$, and in addition $n>\epsilon^{-1}\sqrt{\log(1/\delta)}$, minimizing the above bound with respect to $k$ yields
% \begin{align*}
\eqn{
\E \sbracks{
    F(\ox;\cP)- \min_x F(x;\cP)
}
\lesssim
    \frac{G_0D}{\sqrt{n}}+
    G_{0} D \cdot \bracks{
        \frac{\sqrt{\log(1/\delta)}}{\epsilon n}
    }^{2c/(1+2c)}.
% \end{align*}
}
\end{restatable}
% We refer the reader to Appendix~\ref{app:dpsco_proof} for the full proof.

\begin{remark}
Our result on DP-SCO also recovers the DP-SCO rank-dependent result for learning generalized linear models with convex and Lipschitz losses~\citep{song2021evading}.
\end{remark}

\begin{remark}
When $c>1/2$, $\epsilon = \Theta(1)$ and $\delta = 1/\mathrm{poly}(n)$, the population loss matches the (non-private) informational-theoretical lower bound $\Omega(G_0 D/\sqrt{n})$ \citep{agarwal2009information}.
\end{remark}

\begin{remark}
    % DP-SGD considered in our work randomly queries the (sub)gradient of one function uniformly in each iteration.
    Our results on DP-ERM and DP-SCO naturally generalize to (full-batch) DP-GD.
\end{remark}

\subsection{Overview of Proof Techniques}\label{sec:proof_technique}
The privacy guarantees in Theorems \ref{thm:DPERM} and \ref{thm:DPSCO} follow from Lemma~\ref{lm:privacy_guarantee}.
It suffices to prove the utility guarantees. 
We give an outline of the main proof techniques first and present full proofs afterwards.
The following is a sketch of the core technique for deriving~\eqref{eq:erm_fast_decaying} in Theorem~\ref{thm:DPERM}.
For simplicity, we write $f_j(\cdot)$ for $f(\cdot;s_j)$ and $F(\cdot)$ for $F(\cdot;\cD)$ when there is no ambiguity.

By convexity, the error term of SGD is upper bounded as follows
\eq{
\label{eq:proof_sketch}
    f_{j}(x^{(t)})-f_{j}(x^{*})\leq \nabla f_{j}(x^{(t)})^{\top}(x^{(t)}-x^{*}),
}
where $j \in [n]$ is the random index sampled at iteration $t$.

By definition of $G_{k}$, we know that there is a $k$ dimensional
subspace $U$ such that the gradient component orthogonal to $U$ is small when $G_k$ is small.
Na\"ively, one decomposes the gradient $\nabla f_{j}(x^{(t)})=g_{1}+g_{2}$,
where $g_{1}\in U$ and $g_{2}\in U^{\perp}$,
and separately bounds the two terms $g_1^\top (x^{(t)}-x^{*})$ and $g_2^\top (x^{(t)}-x^{*})$.
Since $g_{1}$ lies in a $k$ dimensional subspace, one can follow existing
arguments on DP-SGD to bound $g_{1}^{\top}(x^{(t)}-x^{*})$. 
Unfortunately, this argument does not give a dimension-independent bound.
Although
$\|g_{2}\|_{2}\leq G_{k}$ (which can be small for large $k$), the term
$\|x^{(t)}-x^{*}\|_{2}$ is as large as $\Omega(\sqrt{d})$ with high probability due
to the isotropic Gaussian noise injected in DP-SGD. 

% {[}Yintat:Discuss why projected DP-SGD does not work. I am not super sure. But if you have a good explanation, put it here.{]}
% \Daogao{I write the following paragraph. But i feel it is better to put it somewhere else like related work}
% One natural way to derive a good bound on $\|x^{{(t)}}-x^{*}\|_2$ is through projection.
% However, previous lower bounds in constrained optimization already suggest that DP projected-SGD cannot evade the curse of dimensionality~\cite{bassily2014private}, since projections tend to destroy the low-rank structure.
% The unconstrained low-rank DP-ERM considered in \cite{song2021evading} is in some sense a rank-dimensional problem, say $\nabla f_{j}(x)\in U$ and adding noise on the null space $U^{\top}$ does not influence the excess empirical risk. Doing projection, however, may lead to a larger loss and destroy the low-rank structure in spirit.

% Our key idea is to partition the whole space $\R^d$ into  $\lfloor \log(d/k) \rfloor + 2$ orthogonal sub-spaces, expressing the error term $\nabla f_j(x^{(t)})^\top (x^{(t)}-x^*)$ as the sum of terms where each term corresponds to the projection to a particular subspace.
Our key idea is to partition the whole space $\R^d$ into  $\lfloor \log(d/k) \rfloor + 2$ orthogonal subspaces, expressing the error term $\nabla f_j(x^{(t)})^\top (x^{(t)}-x^*)$ as the sum of individual terms, each of which corresponds to a projection to a particular subspace.
Fix a $k \in [d]$, and consider the following subspaces:
Let $U_{0}=\text{range}(P_{k})$,
 $U_{s}$ be the subspace orthogonal to all previous subspaces such that $\bigoplus_{i=0}^s U_i = \text{range}(P_{2^sk})$ for $s=1,2,\cdots,\left\lfloor \log(d/k)\right\rfloor $,
and $U_{S}$ be the subspace such that the orthogonal direct sum of all subspaces $\{U_i\}_{i=0}^S$ is $\R^d$, where $S=\left\lfloor \log(d/k)\right\rfloor +1$.
Here, $P_{i}$ is the orthogonal projection matrix with rank $i$ promised by $G_{i}$ in Assumption~\ref{assm:G_k}. 
Let $Q_{s}$ be the orthogonal projection to the subspace $U_{s}$, and observe that $\rank (Q_{s}) \leq 2^{s}k$ and
$\|Q_{s}\nabla F(x)\|_{2}\leq G_{2^{s-1}k}\text{ for all }x\text{ and all }s\geq1.$
Rewriting the right hand side of~\eqref{eq:proof_sketch} with this decomposition yields
$$
f_{j}(x^{(t)})-f_{j}(x^{*})  \leq \lp Q_{0}\nabla f_{j}(x^{(t)})+\sum_{s=1}^{S} Q_{s}\nabla f_{j}(x^{(t)})\rp^{\top}(x^{(t)}-x^{*}).    
$$
On the one hand, if $G_k$ decays quickly, $\| \mathbb{E}_j [  Q_{s} \nabla f_{j} ] \|_2$ can be small for large $s$. 
On the other hand, we expect $\|Q_{s} (x^{(t)}-x^{*})\|_2$ to be small for small $s$ where $Q_{s}$ is an orthogonal projection onto a small subspace. 
Thus, for each $s$, $\nabla f_{j}(x^{(t)})^{\top} Q_{s} (x^{(t)}-x^{*})$ is small either due to a small gradient (small $Q_s \nabla f_{j}$ in expectation over the random index) or small noise (small $Q_{s} (x^{(t)}-x^{*})$), since noise injected in DP-SGD is isotropic. 
More formally, in Lemma~\ref{lem:radius}, we show that for any projection matrix $Q$ with rank $r$, $\|Q(x^{(t)}-x^{(0)})\|_2$ can be upper bounded by a term that depends only on $r$ (rather than $d$).


\subsection{Proof of Theorem~\ref{thm:DPERM}}\label{app:dperm_proof}

Before bounding the utility of DP-SGD, we first bound $x^{(t)}-x^{(0)}$ in expectation.
% \Daogao{Maybe we can move this statement earlier? As we discussed $f_j$ and $\nabla F$ before.}

\begin{lemm}
\label{lem:radius}
Suppose Assumption~\ref{assm:G_k} holds.
Let $Q$ be an orthogonal projection matrix with rank $r$ and suppose that $\|Q\nabla f(x;s)\|_2 \leq G_Q$ for all $x\in\R^d$ and $s \in \mathcal{S}$. 
If we set $\eta\leq\frac{1}{2\alpha}$ in DP-SGD, then for all $t>0$, we have
\[
\E\|Q(x^{(t)}-x^{(0)})\|_2^{2}\leq\frac{4 G_{Q}^{2}}{\alpha^{2}}+\frac{2\eta G_{0}^{2}}{\alpha}(1+r\sigma^{2}).
\]
\end{lemm}


\begin{proof}[Proof of Lemma \ref{lem:radius}]
By the assumption, we know $\|Q \nabla F(x)\|_2\le G_Q$ and $\| \nabla F(x)\|_2\leq G_0$.
Let $z^{(t)}=x^{(t)}-x^{(0)}$. Note that
\begin{align*}
    z^{(t+1)}=& ~ x^{(t+1)}-x^{(0)}\\
    =& ~ x^{(t)}-\eta\Big(\nabla f_j(x^{(t)})+\alpha(x^{(t)}-x^{(0)})+G_{0}\cdot \zeta\Big)-x^{(0)}\\
    =& ~ (1-\alpha\eta)z^{(t)}-\eta(\nabla f_{j}(x^{(t)})+G_{0}\cdot\zeta ),
\end{align*}
where $\zeta\sim \cN(0,\sigma^{2}I_d)$ is the isotropic Gaussian noise drawn in $(t+1)$th step. 
For simplicity, we use $\Tnabla f_j(x^{(t)})$ to denote the noisy subgradient $\nabla f_{j}(x^{(t)})+G_{0}\cdot\zeta$.
Hence, we have
\begin{align*}
    \|Qz^{(t+1)}\|_2^2=& ~ (1-\alpha\eta)^2\|Qz^{(t)}\|_2^2-2\eta(1-\alpha\eta)( Qz^{(t)})^\top Q\Tnabla f_j(x^{(t)})+\eta^2\|Q\Tnabla f_j(x^{(t)})\|_2^2.
\end{align*}

Taking expectation over the random sample $f_j$ and random Gaussian noise $\zeta$, we have
\begin{align*}
\E\|Qz^{(t+1)}\|_{2}^{2}= & (1-\alpha\eta)^{2}\E\|Qz^{(t)}\|_{2}^{2}-2\eta(1-\alpha\eta)\cdot\E\Big((Qz^{(t)})^{\top}(Q\nabla F(x^{(t)}))\Big) \\
 & +\eta^{2}\E\|Q\Tnabla f_j(x^{(t)})\|_{2}^{2}\\
\leq & (1-\alpha\eta)\E[\|Qz^{(t)}\|_{2}^{2}]+2 \eta G_{Q}\cdot\E[\|Qz^{(t)}\|_{2}]+\eta^{2}G_{0}^{2}(1+r\sigma^{2}),
\end{align*}
where we used the fact that $\zeta$ has zero mean, $\|\nabla f_{j}(x^{(t)})\|_{2}\leq G_{0}$, $\|Q\nabla F(x^{(t)})\|_{2}\leq G_{Q}$ and $\eta\leq\frac{1}{2\alpha}$. 
Further simplifying and taking expectation over all iterations, we have
\begin{align}
\label{eq:recursion_distance}
\E\|Qz^{(t+1)}\|_{2}^{2} & \leq(1-\alpha\eta)\E\|Qz^{(t)}\|_{2}^{2}+2\eta(\frac{\alpha}{4}\E\|Qz^{(t)}\|_{2}^{2}+\frac{1}{\alpha}G_{Q}^{2})+\eta^{2}G_{0}^{2}(1+r\sigma^{2}) \nonumber\\
 & \leq(1-\frac{\alpha\eta}{2})\E\|Qz^{(t)}\|_{2}^{2}+\frac{2 \eta}{\alpha}G_{Q}^{2}+\eta^{2}G_{0}^{2}(1+r\sigma^{2}).
\end{align}
Using that $z^{(0)}=0$, we know $\E\|Qz^{(0)}\|_2^2=0$.
Solving the recursion (Equation~\eqref{eq:recursion_distance}) gives 
\[
\E\|Qz^{(t)}\|_{2}^{2}\leq\frac{2}{\alpha\eta}(\frac{2 \eta}{\alpha}G_{Q}^{2}+\eta^{2}G_{0}^{2}(1+r\sigma^{2}))
\]
for all $t$. 
This concludes the proof.
\end{proof}


Now, we are ready to bound the utility. 
The proof builds upon the standard mirror descent proof.
\begin{lemm}
\label{lm:utility_DPSGD_appendix}
Let $\delta \in (0, \frac{1}{2}]$, and $\epsilon \in (0, 10]$.
Under Assumption~\ref{assm:G_k}, let $x^{(0)}$ be the initial iterate and $x^{*} \in \R^d$ be such that $\|x^{(0)}-x^{*}\|_2\le D$.
For all $k \in [d]$,
setting $T=\Theta(n^2 + d \log^2 d)$,
 $\sigma=\Theta \bracks{ \frac{\sqrt{T\log(1/\delta)}}{n\epsilon} }$, $\eta=\sqrt{\frac{D^{2}}{T\cdot G_{0}^{2}\cdot k\sigma^{2}}}$
and $\alpha=\frac{1}{D}\sqrt{\sum_{s=1}^S s^2 2^{s}G_{2^{s-1}k}^{2}}$, we have
\begin{align*}
\E [F(\overline{x})- F(x^{*})] &~ \lesssim \frac{G_{0}D\sqrt{k\log(1/\delta)}}{\epsilon n}+D\sqrt{\sum_{s=1}^{S} s^2 2^{s}G_{2^{s-1}k}^{2}},
\end{align*}
where $S = \lfloor \log(d/k) \rfloor + 1$, $\overline{x}$ is the output of DP-SGD, and the expectation is under the randomness of DP-SGD.

Moreover, if $G_k\leq G_0 k^{-c}$ for each $k$ for some $c>1/2$, and in addition $n > \epsilon^{-1} \sqrt{\log(1/\delta)} $, picking the best $k \in [d]$ for the bound above gives
\begin{align*}
    \E[F(\ox;\cD)-F(x^*;\cD)]\lesssim
        G_{0}D\cdot \bracks{
            \frac{\sqrt{\log(1/\delta)}}{\epsilon n}
        }^{2c/(1+2c)}.
\end{align*}
\end{lemm}

\begin{proof}[Proof of Lemma \ref{lm:utility_DPSGD_appendix}]
The above statement is true for $k = d$ by standard arguments in past work~\citep{bassily2014private,song2021evading}.
Now fix a $k \in \{1, \dots, d - 1\}$.
Our key idea is to split the whole space $\R^d$ into different subspaces.
We define the following set of subspaces:
% \begin{itemize}
% \item Let $U_{0}=\text{range}(P_{k})$,
% \item Let $U_{s}=\text{range}(P_{2^{s}k})\backslash U_{s-1}$, for $s=1,2,\cdots,\left\lfloor \log(d/k)\right\rfloor $,
% \item Let $U_{S}=\R^{d}\backslash U_{S-1}$, where $S=\left\lfloor \log(d/k)\right\rfloor +1$,
% \end{itemize}
\begin{itemize}
\item $U_{0}=\text{range}(P_{k})$.
\item For $s=1,2,\ldots,\left\lfloor \log(d/k)\right\rfloor$, 
    let $U_{s} \subseteq \text{range}(P_{2^{s}k})$ be a subspace with maximal dimension such that $U_s \bot U_i$ for all $i=0, \dots, s-1$.
    % Note that $U_s$ having maximal dimension implies that $\bigoplus_{i=0}^{s} U_i \supseteq \text{range}(P_{2^sk})$.
\item For $S=\left\lfloor \log(d/k)\right\rfloor +1$, 
    let $U_{S} \subseteq \R^{d}$ be the subspace such that $\bigoplus_{i=0}^S U_i = \R^d$, and $U_S \bot U_i$ for all $i=0, \dots, S-1$.
\end{itemize}
Recall $P_{i}$ is the orthogonal projection matrix with rank $i$ that gives rise to $G_{i}$ in Assumption~\ref{assm:G_k}. 
In the above, we have assumed that the base of $\log$ is 2.
Let $Q_{s}$ be the orthogonal projection matrix that projects vectors onto the subspace $U_{s}$. 
Note that $\rank(Q_{s})\leq 2^{s}k$ since $U_s \subseteq \text{range}(P_{2^s k})$.
Moreover, it's clear that $U_s \bot \text{range}(P_{2^{s-1}k})$ for all $s \in \{1, \dots, S\}$. 
This is true by construction $\bigoplus_{i=0}^{s-1} U_i \supseteq \text{range}(P_{2^{s-1}k})$ and that 
$U_s \bot \bigoplus_{i=0}^{s-1} U_i$. Thus,
\begin{equation}
\label{eq:Qbound}
\|Q_{s}\nabla F(x)\|_{2} =
\|Q_{s} (I - P_{2^{s-1}k}) \nabla F(x)\|_{2}
\le 
\normop{Q_{s}} \normtwo{(I - P_{2^{s-1}k}) \nabla F(x)}
\leq G_{2^{s-1}k}
\end{equation}
for all $x \in \R^d$ and all $s \in \{1, \dots, S\}$.

Let $j \in [n]$ be the (uniformly random) index sampled in iteration $t$ of DP-SGD.
By convexity of the individual loss $f_j$,
\begin{align*}
    f_{j}(x^{(t)})-f_{j}(x^{*})\leq\nabla f_{j}(x^{(t)})^{\top}(x^{(t)}-x^{*}).
\end{align*}
By construction, $\R^d$ is the orthogonal direct sum of the subspaces $\{U_j\}_{j=0}^S$, and thus any vector $v\in \R^d$ can be rewritten as the sum $\sum_{i=0}^S Q_i v $.
We thus split the right hand side of the above as follows
\begin{align}
f_{j}(x^{(t)})-f_{j}(x^{*}) & \leq \lp Q_{0}\nabla f_{j}(x^{(t)})+\sum_{s=1}^{S}Q_{s}\nabla f_{j}(x^{(t)})\rp^{\top}(x^{(t)}-x^{*}).\label{eq:mirror_two_term}
\end{align}

We use different approaches to bound $(Q_0\nabla f_j(x^{(t)}))^\top (x^{(t)}-x^*)$ and $(Q_s\nabla f_j(x^{(t)}))^\top (x^{(t)}-x^*)$ when $s\geq 1$, and we discuss them separately in the following.

\medskip
{\bf  Bounding  $(Q_0\nabla f_j(x^{(t)}))^\top (x^{(t)}-x^*)$:} Recall that
\begin{align*}
    x^{(t+1)}=x^{(t)}-\eta\Big(\nabla f_j(x^{(t)})+\alpha(x^{(t)}-x^{(0)})+G_{0}\cdot \zeta\Big)
\end{align*}
for some Gaussian $\zeta\sim \cN(0,\sigma^{2}I_d)$. 
Hence, we have
\begin{align}
\label{eq:duality_Q_0}
    %&~~ (\nabla f_{j}(x^{(t)}))^{\top}Q_{0}(x^{(t)}-x^{*})\nonumber \\
    (\nabla f_{j}(x^{(t)}))^{\top}Q_{0}(x^{(t)}-x^{*}) = &~~ \left(\frac{1}{\eta}(x^{(t)}-x^{(t+1)})-\alpha(x^{(t)}-x^{(0)})-G_{0}\cdot\zeta\right)^{\top}Q_{0}(x^{(t)}-x^{*})\nonumber \\
= & ~~ \left(\frac{1}{\eta}Q_0(x^{(t)}-x^{(t+1)})\right)^{\top}Q_{0}(x^{(t)}-x^{*})-\left(
\alpha(x^{(t)}-x^{(0)})+G_{0}\cdot\zeta
\right)^\top Q_0(x^{(t)}-x^*)\nonumber \\
=& \frac{1}{2\eta}
    \bracks{
        \|Q_{0}(x^{(t)}-x^{*})\|_2^{2} - 
        \|Q_{0}(x^{(t+1)}-x^{*})\|_2^{2} + 
        \|Q_{0}(x^{(t)}-x^{(t+1)})\|_2^{2}
    }
\nonumber \\
 ~& -\lp\alpha(x^{(t)}-x^{(0)})+G_{0}\cdot\zeta\rp^{\top}Q_{0}(x^{(t)}-x^{*}),
\end{align}
where we used the fact that $Q_0^2 v = Q_0 v$ for any $v \in \R^d$ (since $Q_0$ is a projection matrix), and the last equality follows from 
\begin{align*}
   2(Q_0(x^{(t)}-x^{(t+1)}))^\top Q_0(x^{(t)}-x^*) \\
 = & \|Q_{0}(x^{(t)}-x^{*})\|_2^{2} - 
    \|Q_{0}(x^{(t+1)}-x^{*})\|_2^{2} + 
    \|Q_{0}(x^{(t)}-x^{(t+1)})\|_2^{2}.
\end{align*}

Taking expectation on $\zeta$ over both sides of Equation~\eqref{eq:duality_Q_0} and making use of the fact that $\zeta$ has mean $0$, we have
\begin{align*}
\E_{\zeta}(Q_{0}\nabla f_{j}(x^{(t)}))^{\top}(x^{(t)}-x^{*}) = & \frac{1}{2\eta}
    \lp\E_{\zeta}\|Q_{0}(x^{(t)}-x^{*})\|_2^{2} - 
        \E_{\zeta}\|Q_{0}(x^{(t+1)}-x^{*})\|_2^{2} + 
        \E_{\zeta}\|Q_{0}(x^{(t)}-x^{(t+1)})\|_2^{2}\rp\\
 & -\alpha\E_{\zeta}\lp (x^{(t)}-x^{(0)})^{\top}Q_{0}(x^{(t)}-x^{*}) \rp.
\end{align*}

Recalling the definition of $Q_0$ and that $Q_0$ has rank at most $k$, one has
\begin{align*}
\E_{\zeta}\|Q_{0}(x^{(t)}-x^{(t+1)})\|_2^{2} =
    &~ \eta^{2}\E_{\zeta}\|Q_{0}\big(\nabla f_{j}(x^{(t)})+\alpha(x^{(t)}-x^{(0)})+G_{0}\cdot\zeta\big)\|_2^{2}\\
    = &~ \eta^{2}\E_{\zeta}\|Q_{0}\big(\nabla f_{j}(x^{(t)})+\alpha(x^{(t)}-x^{(0)})\big)\|_2^{2}+\eta^{2}G_{0}^{2}k\sigma^{2}\\
\leq &~
    2\eta^{2}G_{0}^{2}(1+k\sigma^{2}) + 
    2\eta^{2}\alpha^{2}\E_{\zeta}\|Q_{0}(x^{(t)}-x^{(0)})\|_2^{2}.
\end{align*}

Moreover, one has
\begin{align*}
-\alpha(x^{(t)}-x^{(0)})^{\top}Q_{0}(x^{(t)}-x^{*}) = &~ -\alpha(x^{(t)}-x^{(0)})^{\top}Q_{0}(x^{(t)}-x^{(0)})-\alpha(x^{(t)}-x^{(0)})^{\top}Q_{0}(x^{(0)}-x^{*})\\
\leq &~
    -\frac{\alpha}{2}\|Q_{0}(x^{(t)}-x^{(0)})\|_2^{2} + 
    \frac{\alpha}{2}\|Q_{0}(x^{(0)}-x^{*})\|_2^{2}.
\end{align*}

Therefore, we have
\begin{align}
\E_{\zeta}(Q_{0}\nabla f_{j}(x^{(t)}))^{\top}(x^{(t)}-x^{*}) \leq & ~ \frac{1}{2\eta}
        \lp\E_{\zeta}\|Q_{0}(x^{(t)}-x^{*})\|_2^{2} - 
           \E_{\zeta}\|Q_{0}(x^{(t+1)}-x^{*})\|_2^{2}
        \rp + \eta G_0^2(1+k\sigma^2) \nonumber\\ 
     &~~~~ +\eta \alpha^2\E_{\zeta}\| Q_0(x^{(t)}-x^{(0)})\|_2^2 - 
                \frac{\alpha}{2}\E_{\zeta}\|Q_{0}(x^{(t)}-x^{(0)})\|_2^{2} + 
                \frac{\alpha}{2}\E_{\zeta}\|Q_{0}(x^{(0)}-x^{*})\|_2^{2}  \nonumber \\
     \leq & ~ \frac{1}{2\eta} \lp\E_{\zeta}\|Q_{0}(x^{(t)}-x^{*})\|_2^{2} - 
                \E_{\zeta}\|Q_{0}(x^{(t+1)}-x^{*})\|_2^{2}  \rp  \\
                &~~~~+\eta G_0^2(1+k\sigma^2)+\frac{\alpha}{2}\E_{\zeta}\|Q_0(x^{(0)}-x^*)\|_2^2 \label{eq:term1},
\end{align}
where we used $\eta\leq\frac{1}{2\alpha}$ at the end. 
\medskip

{\bf  Bounding $(Q_s\nabla f_j(x^{(t)}))^\top (x^{(t)}-x^*)$ :}
We bound the objective above for each $s$ separately. 
By taking expectation over the random $f_j$, we have
\begin{align}
\E_{f_j}(Q_{s}\nabla f_{j}(x^{(t)}))^{\top}(x^{(t)}-x^{*}) = &~ (Q_{s}\nabla F(x^{(t)}))^{\top}(x^{(t)}-x^{*}) \nonumber\\
\leq & ~ \|Q_{s}\nabla F(x^{(t)})\|_2
            \cdot \|Q_s(x^{(t)}-x^{*})\|_2 \nonumber \\
\leq &~ \frac{1}{\alpha_{s}}\|Q_{s}\nabla F(x^{(t)})\|_2^{2} + 
            \frac{\alpha_{s}}{4}\|Q_{s}(x^{(t)}-x^{*})\|_2^{2} \nonumber \\
\leq &~ \frac{G_{2^{s-1}k}^{2}}{\alpha_{s}} + 
            \frac{\alpha_{s}}{2}\|Q_{s}(x^{(t)}-x^{(0)})\|_2^{2} +
            \frac{\alpha_{s}}{2}\|Q_{s}(x^{(0)}-x^{*})\|_2^{2} \label{eq:term2},
\end{align}
where we chose $\alpha_{s}=\alpha s^{-2} 2^{-s}$ and used the bound \eqref{eq:Qbound} and Young's inequality at the end.

\medskip

{\bf Bounding Equation \eqref{eq:mirror_two_term}:} Combining both the terms \eqref{eq:term1} and \eqref{eq:term2} and taking expectation
over all randomness, we have
\begin{align*}
\E [F(x^{(t)})- F(x^{*})] \leq &~~ \frac{1}{2\eta}(
    \E\|Q_{0}(x^{(t)}-x^{*})\|_2^{2} - 
    \E\|Q_{0}(x^{(t+1)}-x^{*})\|_2^{2}) +
    \eta G_{0}^{2}(1+k\sigma^{2})+\frac{\alpha}{2}\E\|Q_{0}(x^{(0)}-x^{*})\|_2^{2}\\
 & +\sum_{s=1}^{S}\frac{G_{2^{s-1}k}^{2}}{\alpha_{s}} + 
    \frac{1}{2}\sum_{s=1}^{S}\alpha_{s}\E\|Q_{s}(x^{(t)}-x^{(0)})\|_2^{2} + 
    \frac{1}{2}\sum_{s=1}^{S}\alpha_s\E\|Q_{s}(x^{(0)}-x^{*})\|_2^{2}\\
\leq &~ \frac{1}{2\eta}(
    \E\|Q_{0}(x^{(t)}-x^{*})\|_2^{2} - 
    \E\|Q_{0}(x^{(t+1)}-x^{*})\|_2^{2}) +
    \eta G_{0}^{2}(1+k\sigma^{2}) + 
    \frac{\alpha}{2}\|x^{(0)}-x^{*}\|_2^{2}\\
 & +\sum_{s=1}^S\frac{G_{2^{s-1}k}^{2}}{\alpha_{s}} + 
    \frac{1}{2}\sum_{s=1}^S\alpha_{s}\E\|Q_{s}(x^{(t)}-x^{(0)})\|_2^{2}.
\end{align*}

Recall $\alpha\cdot\eta\leq 1/2$. Under the other assumptions, by Lemma \ref{lem:radius}, one can show
\begin{align*}
\E\|Q_{s}(x^{(t)}-x^{(0)})\|^{2} & \leq\frac{4 G_{2^{s-1}k}^{2}}{\alpha^{2}}+\frac{2\eta G_{0}^{2}}{\alpha}(1+2^{s}k\sigma^{2})\\
 & \leq\frac{4 G_{2^{s-1}k}^{2}}{\alpha_{s}^{2}}+\frac{2\eta G_{0}^{2}}{\alpha_{s} s^2}(1+k\sigma^2).
\end{align*}
Using $\sum_{s=1}^{\infty} s^{-2} \leq 2$, we have
\begin{align*}
\E F(x^{(t)})-\E F(x^{*}) \leq &~ \frac{1}{2\eta}(
    \E\|Q_{0}(x^{(t)}-x^{*})\|_2^{2} - 
    \E\|Q_{0}(x^{(t+1)}-x^{*})\|_2^{2}
) + \frac{\alpha}{2}\|x^{(0)}-x^{*}\|^{2}\\
&+ \eta G_0^2(1+k\sigma^2)+3\sum_{s=1}^S\frac{G_{2^{s-1}k}^2}{\alpha_s}+2 \eta G_0^2(1+k\sigma^2)\\
\leq  &~ \frac{1}{2\eta}(
    \E\|Q_{0}(x^{(t)}-x^{*})\|_2^{2} -
    \E\|Q_{0}(x^{(t+1)}-x^{*})\|_2^{2}
) + \frac{\alpha}{2}\|x^{(0)}-x^{*}\|_2^{2}\\
&~ +3\eta G_{0}^{2}(1+k\sigma^{2}) +\frac{3}{\alpha}\sum_{s=1}^S s^2 2^{s}G_{2^{s-1}k}^{2}.
\end{align*}
Summing up over $t=1,2,\cdots,T$, by the assumption that $\|x^{(0)}-x^*\|_2\leq D$ and convexity of the function, we have
\begin{align}
\label{eq:utility_eta_alpha}
\E [F(\ox)- F(x^{*})]\leq\frac{D^{2}}{2\eta T}+3\eta G_{0}^{2}(1+k\sigma^{2})+\frac{\alpha}{2}D^{2}+\frac{3}{\alpha}\sum_{s=1}^S s^2 2^{s}G_{2^{s-1}k}^{2}.    
\end{align}

Set the parameters $T=c_1 (n^2 + d \log^2 d)$, $\sigma=\frac{c_2\sqrt{T\log(1/\delta)}}{ n \epsilon}$,
$\eta=\sqrt{\frac{D^{2}}{T\cdot G_{0}^{2}\cdot k\sigma^{2}}}$
and $\alpha=\frac{1}{D}\sqrt{\sum_{s=1}^{S}s^2 2^{s}G_{2^{s-1}k}^{2}}$ for some large constants $c_1,c_2$. 
Note that this choice of parameters satisfies
\begin{align*}
\eta\cdot\alpha
& =
    \sqrt{\frac{D^{2}}{T\cdot G_{0}^{2}\cdot k\sigma^{2}}}\cdot\frac{\sqrt{\sum_{s=1}^{S}s^2 2^{s}G_{2^{s-1}k}^{2}}}{D}\\
 & \leq
    \sqrt{\frac{ G_{0}^{2} (2d) \log^3 (2d)}{T\cdot G_{0}^{2}\cdot k\sigma^{2}}}
    = 
        \frac{n\epsilon}{c_{2}T}\sqrt{\frac{ (2d) \log^3 (2d) }{k \cdot\log(1/\delta)}}\\
 & \leq\frac{n\epsilon\sqrt{ (2d) \log^3 (2d) }}{c_{2}T}\leq\frac{1}{2},
\end{align*}
where we used the fact that $G_k \leq G_0$, $s \le S \le \log (2d)$ , $T \geq n^2 + d \log^2 d$, and $c_2$ is large enough.

Using the parameters we pick, we have
\[
\E [F(\overline{x})- F(x^{*})]\lesssim \frac{G_{0}D\sqrt{k\log(1/\delta)}}{\epsilon n}+D \sqrt{\sum_{s=1}^{S} s^2 2^{s}G_{2^{s-1}k}^{2}}
\]

Moreover,
assuming $G_{k}\leq G_0 k^{-c}$ for some $c>1/2$, we have $\sqrt{\sum_s s^2 2^sG_{2^{s-1}k}^2}\lesssim G_0 / k^{c}$. 
Hence,
\begin{align*}
    \E [F(\overline{x})- F(x^{*})]\lesssim\frac{G_{0}D\sqrt{k\log(1/\delta)}}{\epsilon n}+\frac{G_0D}{k^c}.
\end{align*}


%For the general case, taking back the value of $T=\Theta(n^2/k)$ and $\sigma=\frac{\sqrt{T\log(1/\delta)}}{n\epsilon}=\frac{\sqrt{\log(1/\delta)}}{\epsilon\sqrt{k}}$. 
%If $\sqrt{\sum_{s}2^{s}G_{2^{s-1}k}^{2}}\leq \frac{G_0n\sqrt{\log d\log(1/\delta)}}{2\epsilon\sqrt{k}}$ and choose the best value of $\eta=\frac{D\epsilon\sqrt{k} }{G_0n\sqrt{\log d\log(1/\delta)}}$ and $\alpha=\frac{\sqrt{\sum_{s}2^{s}G_{2^{s-1}k}^{2}}}{D}$,
%we have
%\begin{align*}
%    \E [F(\overline{x})- F(x^{*})]\lesssim\frac{G_{0}D\sqrt{k\log d\log(1/%\delta)}}{\epsilon n}+D\cdot\sqrt{\sum_{s}2^{s}G_{2^{s-1}k}^{2}}.
%\end{align*}

%Moreover,
%assuming $G_{k}\leq\frac{G_{0}}{\epsilon\sqrt{k}}$, we have $\sqrt{\sum_s2^sG_{2^{s-1}k}^2}\leq \frac{G_0\sqrt{\log d}}{\epsilon \sqrt{k}}$ and thus satisfy the condition that $\sqrt{\sum_{s}2^{s}G_{2^{s-1}k}^{2}}\leq \frac{G_0n\sqrt{\log d\log(1/\delta)}}{2\epsilon\sqrt{k}}$.
%Choosing the parameters according to the general case, we have
%\begin{align*}
%    \E [F(\overline{x})- F(x^{*})]\lesssim\frac{G_{0}D\sqrt{k\log d\log(1/\delta)}}{\epsilon n}+\frac{G_0D\sqrt{\log d}}{\epsilon \sqrt{k}}.
%\end{align*}

Since the above bound holds for all $k \in \{1, \dots, d\}$, we may optimize it with respect to $k$.
Recall by assumption that $ n\geq \epsilon^{-1} \sqrt{\log(1/\delta)}$. 
Letting
\eqn{
    k = \min \left\{ 
        d,  
        \left\lceil \left( \frac{\epsilon n}{\sqrt{\log(1/\delta)}} \right)^{\frac{2}{1+2c}} \right\rceil
    \right\}
}
yields the bound
\eqn{
    \E[F(\ox;\cD)-F(x^*;\cD)]\lesssim G_{0}D \cdot 
        \bracks{
            \frac{\sqrt{\log(1/\delta)}}{\epsilon n}
        }^{2c/(1+2c)}.
}
\end{proof}

Combining the privacy guarantee in Lemma~\ref{lm:privacy_guarantee} and Lemma~\ref{lm:utility_DPSGD_appendix} directly results in Theorem~\ref{thm:DPERM}.

\subsection{Proof of Theorem~\ref{thm:DPSCO}} \label{app:dpsco_proof}

We study the generalization error of DP-SGD and make use of its stability.
The bound on the excess population loss follows from combining bounds on the excess empirical loss and the generalization error. 
Before stating the proof, we first recall two results in the literature.

\begin{lemm}[{\cite[Lemma 7]{BE02}}]
\label{lm:stab_generalization_error}
Given a learning algorithm $\cA$, a dataset $\cD=\{s_1,\cdots,s_n\}$ formed by $n$ i.i.d. samples drawn from the underlying distribution $\cP$, and we replace one random sample in $\cD$ with a freshly sampled $s'\sim\cP$ to obtain a new neighboring dataset $\cD'$.
One has
\begin{align*}
    \E_{\cD,\cA}[F(\cA(\cD); \cP)-F(\cA(\cD);\cD)]=\E_{\cD,s'\sim\cP,\cA}[f(\cA(\cD);s')-f(\cA(\cD');s')],
\end{align*}
where $\cA(\cD)$ is the output of $\cA$ with input $\cD$.
\end{lemm}

\begin{lemm}[{\cite[Theorem 3.3]{BFGT20}}]
\label{lm:stab_DPSGD}
Suppose Assumption~\ref{assm:G_k} holds, running DP-SGD with step size $\eta$ 
on any two neighboring datasets $\cD$ and $\cD'$ for $T$ steps yields the following bound
\begin{align*}
    \E \sbracks{ \|\ox-\ox'\|_2 } \le 4 G_0 \eta\left(\frac{T}{n}+\sqrt{T}\right),
\end{align*}
where $\ox$ and $\ox'$ are the outputs of DP-SGD with datasets $\cD$ and $\cD'$, respectively.
\end{lemm}

% Combining our previous results, our result on DP-SCO follows.

% \DPSCO*

% We now state the proof of Theorem~\ref{thm:DPSCO}.

\begin{proof}[Proof of Theorem \ref{thm:DPSCO}]
Let $\ox$ and $\ox'$ be the outputs of DP-SGD when applied to the datasets $\cD$ and $\cD'$, respectively.
$\cD'$ is a neighbor of $\cD$ with one example replaced by $s' \sim \cP$ that is independently sampled. 
Combining Lemma~\ref{lm:stab_generalization_error} and Lemma~\ref{lm:stab_DPSGD} yields
\begin{align*}
    \E [F(\ox; \cP)-F(\ox;\cD)]= &~ \E[f(\ox;s')-f(\ox';s')]\\
    \leq & ~\E [G_0 \|\ox-\ox'\|_2]\\
    \leq & ~ 4G_0^2\eta \bracks{ \frac{T}{n}+\sqrt{T}}.
\end{align*}

Similar to the DP-ERM case, 
by setting $T=c_1 (n^2 + d \log^2 d)$,
$\sigma=\frac{c_2\sqrt{T\log(1/\delta)}}{n\epsilon}$, 
$\eta=\sqrt{\frac{D^{2}}{T \cdot G_{0}^{2}(T/n+k\sigma^{2})}}$
and $\alpha=\frac{1}{D} \sqrt{\sum_{s=1}^S s^2 2^{s}G_{2^{s-1}k}^{2}}$ for some large positive constants $c_1$ and $c_2$, we conclude that $\eta\cdot\alpha\leq 1/2$. Hence, Equation~\eqref{eq:utility_eta_alpha} shows that, for any fixed dataset $\cD$ and any $x^*$ such that $\normtwo{x^{(0)} - x^*} \le D$, we have
\begin{align*}
\E [F(\ox;\cD)- F(x^{*};\cD)]\leq\frac{D^{2}}{2\eta T}+3\eta G_{0}^{2}(1+k\sigma^{2})+\frac{\alpha}{2}D^{2}+\frac{3}{\alpha}\sum_{s=1}^S s^2 2^{s}G_{2^{s-1}k}^{2}.    
\end{align*}

We can rewrite the population loss as follows
\begin{align*}
\E[F(\ox; \cP)-F(x^*; \cP)] =& ~ \E[F(\ox; \cP)-F(\ox;\cD)]+\E_{\cD}[F(\ox;\cD)-F(x^*;\cD)]\\
\leq &~ 
    4G_0^2\eta \bracks{ \frac{T}{n}+\sqrt{T} } + 
    \frac{D^{2}}{2\eta T} + 
    3\eta G_{0}^{2}(1+k\sigma^{2}) + 
    \frac{\alpha}{2}D^{2} + 
    \frac{3}{\alpha}\sum_{s=1}^S s^2 2^{s}G_{2^{s-1}k}^{2}.
\end{align*}

Substituting in the values for parameters $T$, $\sigma$, $\eta$, and $\alpha$ yields
\begin{align*}
    \E[F(\ox; \cP)-F(x^*; \cP)]\lesssim \frac{G_0D}{\sqrt{n}}+\frac{G_{0}D\sqrt{k\log(1/\delta)}}{\epsilon n}+D\sqrt{\sum_{s=1}^{S}s^2 2^{s}G_{2^{s-1}k}^{2}} 
\end{align*}
for all $k \in [d]$.

Similarly, if we have $G_k\leq G_0 k^{-c}$ for some $c>1/2$, and in addition $n > \epsilon^{-1} \log(1/\delta)$, it immediately follows that
\begin{align*}
    \E[F(\ox; \cP)-\min_x F(x; \cP)]\lesssim \frac{G_0D}{\sqrt{n}}+G_{0}D\cdot \bracks { \frac{\sqrt{\log(1/\delta)}}{\epsilon n} }^{2c/(1+2c)}.
\end{align*}
This completes the proof.
\end{proof}