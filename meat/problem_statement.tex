\section{Problem Statement}
We build models for sentence classification and language generation tasks with datasets of modest sizes under (central/global) approximate-DP (also known as $(\epsilon, \delta)$-DP)~\citep{dwork2014algorithmic}. 
\begin{defi}[$(\epsilon, \delta)$-DP]
A randomized algorithm $\M: \X \to \Y$ is ($\epsilon, \delta$)-differentially private if for all adjacent datasets $X, X'\in\mathcal{X}$ and all $Y \subset \Y$, 
$
\Prob{\M(X) \in Y} \le \exp( \epsilon ) \Prob{ \M(X') \in Y } + \delta.
\label{eq:epsilon_delta_dp}
$
\end{defi}
We say that two datasets are adjacent if and only if one can be obtained from the other by including an extra record.\footnote{An alternative definition of adjacency assumes all datasets are of equal size and is based on the replacement of records. This is not the definition we adopt. }
How a record is defined is task dependent and will be made clear below.

Intuitively, DP algorithms ensure that random outputs obtained from similar inputs are difficult to distinguish. 
$\epsilon$ and $\delta$ are \textit{privacy leakage} parameters that measure the loss of privacy and small values imply stronger privacy guarantees.
Unlike heuristic privacy notions~\citep{huang2020instahide}, DP allows for the tracking of privacy loss through the calculation of leakage parameters, and ensures privacy under composition~\citep{dwork2014algorithmic}, meaning that the overall privacy loss of multiple DP algorithms releasing multiple statistics can be reasoned in a principled manner.

DP learning typically relies on DP optimizers which privatize gradients before performing updates. 
The privatization step ensures that parameter updates leak limited information about training examples through their gradients.
Specifically, this step clips per-example gradients with a norm constraint $C$, and adds Gaussian noise $z\sim\mathcal{N}(0, C^2\sigma^2 I_p)$ to the sum of clipped gradients.
Here, $\sigma$ is the \textit{noise multiplier} determined from the privacy budget $(\epsilon, \delta)$, number of gradient updates $S$, and sampling rate $q=\tfrac{B}{N}$ for a batch size of $B$ and a dataset size of $N$.\footnote{Since we adopted the definition of ``neighboring'' based on addition/removal, the batch size here should be interpreted as the lot size~\citep{abadi2016deep} or, equivalently stated, the expected size of a batch drawn with Poisson sampling.}
Intuitively, clipping individual gradients ensures that each example has bounded influence on the parameter update, whereas noising the gradient prevents exact tracing of particular examples.
The noise being isotropic implies that larger models would experience heavier noise per update, as the norm of the $p$-dimensional Gaussian $\normtwo{z}$ scales as $C \sigma \sqrt{p}$.
This is widely believed to be the cause for DP optimization to perform poorly at training high-dimensional deep learning models~\citep{gautum14,yu2021not}.

Our starting point for building DP language models is (public) pretrained models. 
Pretrained language models tend to contain general knowledge of language~\citep{manning2020emergent} and thus should make the downstream private learning problem easier.
We fine-tune these models with DP-Adam~\citep{abadi2016deep,kingma2014adam} (see Chapter~\ref{sec:dp_first_order} for details) and track privacy loss through R\'enyi DP~\citep{mironov2017renyi}, but also report the converted $\epsilon$ from a Gaussian DP central limit theorem~\citep{dong2019gaussian} and from accurately composing tradeoff functions via fast Fourier transform~\citep{gopi2021numerical}.
We consider privacy levels $\epsilon \in \{3,8\}$ and $\delta = \tfrac{1}{2 |\mathcal{D}_\text{train}|}$ throughout
for a training set of size $|\mathcal{D}_\text{train}|$.
We tune hyperparameters on a text generation task (E2E; introduced below) and transfer these to the remaining tasks.
We outline two broad classes of NLP problems considered in this paper and define what constitutes a record below.

\paragraph{Sentence Classification.}
The goal is to learn a model that classifies sentences into one of a few categories.
For these tasks, each example/record consists of input sentences and a label to be predicted.
We fine-tune models of various sizes in the BERT~\citep{devlin2018bert} and RoBERTa~\citep{liu2019roberta} families, as these models are known to work well in non-private learning.

\paragraph{Language Generation.}
The goal is to learn a model that generates natural language sentences given some context.
For table-to-text generation tasks such as E2E~\citep{novikova2017e2e} and DART~\citep{nan2020dart}, each example/record in the training data consists of a pair of table entry and corresponding text description to be predicted. 
For a dialogue generation task such as Persona-Chat~\citep{zhang2018personalizing}, each example/record consists of metadata, a dialogue history, and a response to be predicted. 
We fine-tune GPT-2~\citep{radford2019language} and variants of different sizes for these problems, as this model family is known to work well for text generation. 
