\appendix
\section{Supplementary for Chapter 3}

\subsection{Derivation of the Frobenius norm identity} \label{app:frobenius}
Recall $a \in \R^{B\times T \times d}$ is the input to a linear layer with weight matrix $W \in \R^{p \times d}$, and $g \in \R^{B \times T \times p}$ is the gradient of the loss w.r.t. the output. The identity follows from trivial algebra:
\eqn{
\normf{ \nabla_W \mathcal{L}_i }^2 = \normf{ g_i^\top a_i }^2
&= \normf{ \sum_{k=1}^T g_{i, k} a_{i, k}^\top }^2 \\
&=
    \sum_{r=1}^d \sum_{s=1}^p \bracks{
        \sum_{k=1}^T a_{i, k, r} g_{i, k, s}
    }^2 \\
&=
    \sum_{r=1}^d \sum_{s=1}^p \sum_{k_1=1}^T \sum_{k_2=1}^T 
        a_{i, k_1, r} g_{i, k_1, s}
        a_{i, k_2, r} g_{i, k_2, s} \\
&=
    \sum_{k_1=1}^T \sum_{k_2=1}^T
        \bracks{ \sum_{r=1}^d a_{i, k_1, r} a_{i, k_2, r} }
        \bracks{ \sum_{s=1}^p g_{i, k_1, s} g_{i, k_2, s} } \\
&=
    \mathrm{vec}(a_{i} a_{i}^\top)^\top \mathrm{vec}( g_i g_i^\top ). 
}
Note that when $T=1$, the identity takes the form of 
\eqn{
\normf{ \nabla_W \mathcal{L}_i }^2
= 
    \mathrm{vec}(a_{i} a_{i}^\top)^\top \mathrm{vec}( g_i g_i^\top )
= 
    \normtwo{a_i}^2 \normtwo{g_i}^2. 
}
This is the \cite{goodfellow2015efficient} trick.

\newpage

\subsection{Experiment Details}\label{app:experiments_main}

\paragraph{Sentence Classification.}
Results for RGP in Table~\ref{table:glue} are taken from documented numbers in their released codebase.\footnote{\url{https://github.com/dayu11/Differentially-Private-Deep-Learning/tree/main/language}}
These results are under the DP guarantees of $(\epsilon, \delta)=(3, 10^{-5})$ or $(\epsilon, \delta)=(8, 10^{-5})$.
These guarantees are strictly looser than our guarantees which are based on $\delta=\nicefrac{1}{2 |\D_\text{train}|}$ (recall the smallest dataset in this cohort of tasks has $60$k+ records).
The RGP numbers in Table~~\ref{table:glue} are higher than those reported in their paper~\citep{yu2021large}, since the latter numbers are not based on fine-tuning the official RoBERTa models.

\paragraph{Table-To-Text Generation.}
To evaluate models trained on E2E and DART, we evaluate generations from models obtained with beam search with a beam size of $5$.
For evaluation, we run the official pipeline for E2E,\footnote{\url{https://github.com/tuetschek/e2e-metrics}} and the pipeline used in the GEM benchmark~\citep{gehrmann2021gem} for DART.\footnote{\url{https://github.com/GEM-benchmark/GEM-metrics}}

\paragraph{Chit-Chat Dialog Generation.}
We built off Huggingface's codebase of the winning entry of the ConvAI2 competition,
\footnote{\url{https://github.com/huggingface/transfer-learning-conv-ai}}
\footnote{\url{http://convai.io/2018/}} and used their preprocessed training set with the minor modification of truncating the number of training examples to be a multiple of the batch size.
The original ConvAI2 competition is aimed at advancing research on building engaging chatbots and also requested models to predict the mostly likely response given a list of candidates.
The challenge included hits@1 as part of its suite of automatic metrics.
For simplicity, we skip this step of predicting the most likely response for both training and evaluation. 
Results for the entry HuggingFace (ConvAI2 winner) in Table~\ref{table:personachat} are taken from the official validation set leader board.\footnote{\url{https://github.com/DeepPavlov/convai/blob/master/leaderboards.md}}
Our reimplementation of HuggingFace's submission uses the released code for their winning entry, fine-tunes GPT with the default hyperparameters, and removes the classification loss for learning to predict the most likely response given candidates.

We additionally fine-tuned DialoGPT-medium~\citep{zhang2019dialogpt}, since the model was pretrained on conversation-like exchanges extracted from Reddit comment chains. Intuitively, this pretraining corpus is more aligned with the downstream fine-tuning data than WebText~\citep{radford2019language}, and thus would likely improve downstream performance.

To evaluate the F1 score, we obtained predicted responses from trained models using beam search with a beam size of $5$.
Since past work found that the F1 score can be gamed by always letting the model predict a predetermined phrase~\citep{dinan2019second}, we additionally ask humans to rate generations from the model through Amazon mechanical turk to obtain a more complete picture of model quality.
When sampling responses for human evaluation, we used nucleus sampling with $p=0.9$~\citep{holtzman2019curious}.
For human evaluation, we asked 20 turkers to each rate 5 entries.
For each entry, a turker is asked to rate on a scale of 1 to 5 the quality of predicted responses from privately fine-tuned DialoGPT-medium models, non-privately fine-tuned DialoGPT-medium models, non-privately fine-tuned GPT models (our reimplementation of HuggingFace's entry), and the reference text, given the history of the dialog. 
Since human evaluation can yield noisy results, we also report the $95\%$ asymptotic confidence interval in Table~\ref{table:personachat}.
All models were trained and evaluated on the version of Persona-Chat with the original persona.
All numbers reported in Table~\ref{table:personachat} are obtained on the validation split. 
