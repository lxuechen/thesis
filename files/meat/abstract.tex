\begin{abstract}









Differentially private deep learning has recently witnessed advances in computational efficiency and privacy-utility trade-off.
We explore whether further improvements along the two axes are possible and provide affirmative answers leveraging two instantiations of \emph{group-wise clipping}. 
To reduce the compute time overhead of private learning, we show that \emph{per-layer clipping}, where the gradient of each neural network layer is clipped separately, allows clipping to be performed in conjunction with backpropagation in differentially private optimization. This results in private learning that is as memory-efficient and almost as fast per training update as non-private learning for many workflows of interest. 
While per-layer clipping with constant thresholds tends to underperform standard flat clipping, per-layer clipping with adaptive thresholds matches or outperforms flat clipping under given training epoch constraints, hence attaining similar or better task performance within less wall time.
To explore the limits of scaling (pretrained) models in differentially private deep learning, we privately fine-tune the 175 billion-parameter GPT-3. 
We bypass scaling challenges associated with clipping gradients that are distributed across multiple devices with \emph{per-device clipping} that clips the gradient of each model piece separately on its host device.
Privately fine-tuning GPT-3 with per-device clipping achieves a task performance at $\epsilon=1$ better than what is attainable by non-privately fine-tuning the largest GPT-2 on a summarization task.




\end{abstract}

