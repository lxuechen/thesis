\section{Experiments}\label{sec:experiment}

Previous sections verified that per-layer clipping has an efficiency advantage over flat clipping. 
We now show that adaptive per-layer clipping is competitive in terms of privacy vs utility.
Our experiments cover training wide ResNets from scratch, fine-tuning RoBERTa on GLUE tasks, and fine-tuning GPT-2 and GPT-3 for table-to-text generation and summarization tasks. For private quantile estimation, we use the geometric update rule by \cite{andrew2019differentially} and set $\eta=0.3$ for all experiments.
Reported numbers are averaged over three seeds unless otherwise stated.



\subsection{CIFAR-10 Classification with WideResNets}\label{subsec:cifar10}

We train a wide ResNet (WRN16-4, 2.8M trainable parameters) \citep{zagoruyko2016wide} from scratch for CIFAR-10 classification with differential privacy. We follow the implementation by  \cite{de2022unlocking}, e.g.,  batch normalization are replaced with group normalization and weight standardization is applied for convolutional layers, except that we do not use augmentation multiplicity for simplicity.
We set privacy parameter $\delta= 10^{-5}$ and choose $\epsilon$ from $\{1,3,5,8\}$, which are typical privacy parameters used in previous works.

We compare the performance of adaptive per-layer clipping with that of flat clipping. 
For both algorithms, we use  hyperparameters suggested by \cite{de2022unlocking}  and tune learning rates. %
We use a fraction $r=0.01$ of privacy budget for quantile estimation and choose the target quantile $q$ from $\{0.5, 0.6, 0.7\}$. For both algorithms we train for 300 epochs. 
% We summarize the details in Appendix \ref{app:cifar}.  
Table~\ref{table:cifar_acc} shows that adaptive per-layer clipping achieves  training and validation accuracies on par with flat clipping for multiple choices of $\epsilon$. %


\begin{table}[th]
\newcommand{\bb}[1]{\textbf{#1}}
\footnotesize
\setlength\tabcolsep{2.4pt}
\caption{Adaptive per-layer clipping achieves  accuracy (in \%) on par with flat clipping for CIFAR-10.}
\centering
\begin{tabular}{c ccc ccc ccc ccc}
\toprule
\multirow{2}[2]{*}{Method} & 
\multicolumn{2}{c}{\text{$\epsilon=1$}} & 
\multicolumn{2}{c}{\text{$\epsilon=3$}} & 
\multicolumn{2}{c}{\text{$\epsilon=5$}} & 
\multicolumn{2}{c}{\text{$\epsilon=8$}} \\
\cmidrule(lr){2-3} \cmidrule(lr){4-5} \cmidrule(lr){6-7} \cmidrule(lr){8-9}
& Train & Valid & Train & Valid & Train & Valid & Train & Valid \\
\midrule
Flat clipping \citep{de2022unlocking} & 44.9 & 44.7 & 65.2 & 63.1 & 72.3 & 70.1 & 77.1 & 73.9 \\
Adaptive per-layer & 49.1 & 48.6 & 65.5 & 63.7 & 72.3 & 69.5 & 77.7 & 73.5 \\ %
\bottomrule
\end{tabular}
\label{table:cifar_acc}
\end{table}


\subsection{Text Classification Tasks with RoBERTa} \label{subsec:roberta}

Our first experiment aims to show that adaptive per-layer clipping is broadly competitive with existing approaches in the literature in terms of the privacy-utility tradeoff. 
In this experiment, we fine-tune RoBERTa-base (125M) and RoBERTa-large (355M)~\citep{liu2019roberta} on SST-2, QNLI, QQP, and MNLI from the GLUE benchmark~\citep{wang2018glue} with differential privacy. 
We set $\epsilon\in \{3,8\}$ and $\delta=1/n^{1.1}$, where $n$ is the size of training set.
We tune the learning rate, batch size, and target quantile on SST-2's training data and transfer the best hyperparameters to other tasks. 
We use $r=0.1$ of the privacy budget for quantile estimation, choose the target quantile $q$ from $\{0.5, 0.75, 0.85\}$, and set the number of training epochs $E=20$.  %
Table~\ref{table:glue_main} shows that adaptive per-layer clipping obtains accuracies competitive with established approaches in the literature under fixed privacy constraints.

Our second controlled experiment shows that adaptive per-layer clipping gives utility that is competitive with flat clipping under fixed training epochs (when both approaches fine-tune the same set of parameters), effectively verifying its wall time advantage. 
In this experiment, we constrain the number of training epochs $E$ to be one of $\{3, 10, 20, 30\}$ and fine-tune RoBERTa models on SST-2 with the two clipping methods.
Table~\ref{table:glue_epoch_sst_2} shows that adaptive per-layer clipping is 
on par with flat clipping in accuracy for all setups and justifies the wall time advantage claim given that adaptive per-layer clipping is faster per epoch. 

\cite{li2022large} demonstrated that the performance of private fine-tuning for text classification with various algorithms improves with a text infilling formulation. 
The infilling technique reformulates the optimization problem and is orthogonal to the algorithmic aspects under study. 
To ensure our comparisons are fair, all experiments in this section follow the usual BERT fine-tuning setup without text infilling. 
% Additional details of the two experiments can be found in Appendix \ref{app:glue}.

\begin{table}[th]


\newcommand{\bb}[1]{\textbf{#1}}

\footnotesize
\setlength\tabcolsep{1.7pt}
\caption{
Adaptive per-layer clipping matches accuracy (in \%) results in the literature on GLUE tasks.\protect\footnotemark
}
\centering
\begin{tabular}{l c cccc cccc}
\toprule
\multirow{2}[2]{*}{Model} &
\multirow{2}[2]{*}{Method} & 
\multicolumn{4}{c}{\text{$\epsilon=3$}} & 
\multicolumn{4}{c}{\text{$\epsilon=8$}} \\
\cmidrule(lr){3-6} \cmidrule(lr){7-10}
& & MNLI-(m/mm) & QQP & QNLI & SST-2
& MNLI-(m/mm) & QQP & QNLI & SST-2 \\
\midrule
\multirow{ 4}{*}{RoBERTa-base} & \cite{yu2021large} & - & - & - & - & 80.1 & 85.5 & 87.2 & 91.6 \\
                               & \cite{li2022large} & $82.47/82.10$ & $85.41$ & $84.62$ & $86.12$ & $83.30/83.13$ & $86.15$ & $84.81$ & $85.89$ \\
                               & \cite{yu2022differentially} & - & - & - & - & $83.5$ & $85.7$ & $87.3$ & $92.2$ \\
                                & Adaptive per-layer & $82.83/83.27$ & $85.67$ & $86.13$ & $92.03$ & $83.70/83.97$ & $86.23$ & $87.13$ & $92.40$ \\
\midrule
\multirow{ 4}{*}{RoBERTa-large} & \cite{yu2021large} & - & - & - & - & 86.1 & 86.7 & 90.0 & 93.0 \\
                                &  \cite{li2022large} & $85.53/85.81$ & $86.65$ & $88.94$ & $90.71$ & $86.28/86.54$ & $87.49$ & $89.42$ & $90.94$ \\
                                & \cite{yu2022differentially} & - & - & - & - & $87.8$	& $87.4$ & $90.8$ & $95.3$ \\
                                & Adaptive per-layer & $87.10/87.20$ & $86.80$ & $89.80$ & $93.87$ & $87.67/87.57$ & $87.20$ & $90.77$ & $94.03$ \\
\bottomrule
\end{tabular}
\label{table:glue_main}
\end{table}
\footnotetext{Results in \cite{yu2021large,yu2022differentially} on MNLI are the average of the matched and mismatched accuracy.}




\begin{table}[th]
\centering
\newcommand{\bb}[1]{\textbf{#1}}
\footnotesize
\setlength\tabcolsep{2.4pt}
\caption{Adaptive per-layer clipping is competitive with flat clipping on SST-2 in accuracy (\%) under fixed fine-tuning epochs ($E$). Models are full fine-tuned.
Numbers in parentheses are standard deviations from three independent runs. 
See results for $\epsilon=8$ in Appendix~\ref{app:glue_controlled_complementary}.
}
\begin{tabular}{l c cccc}
\toprule
\multirow{2}[2]{*}{Model} &
\multirow{2}[2]{*}{Method} & 
\multicolumn{4}{c}{\text{$\epsilon=3$}} \\
\cmidrule(lr){3-6}
& & $E=3$ & $E=10$ & $E=20$ & $E=30$  \\
\midrule
\multirow{2}{*}{RoBERTa-base} %
& Flat clipping (tuned) & $88.70 (0.52)$ & $90.17 (1.10)$ & $91.47 (1.07)$ & $91.60 (0.95)$\\
& Adaptive per-layer & $90.50 (1.21)$ & $91.90 (0.72)$ & $92.33 (0.42)$ & $92.23 (0.06)$ \\
\midrule
\multirow{2}{*}{RoBERTa-large} & Flat clipping (tuned) & $92.20 (0.26)$ & $93.07 (0.75)$ & $93.67 (0.40)$ & $94.23 (0.67)$ \\
& Adaptive per-layer & $91.73 (0.59)$ & $93.27 (0.45)$ & $93.90 (0.26)$ & $94.13 (0.38)$ \\
\bottomrule
\end{tabular}
\label{table:glue_epoch_sst_2}
\end{table}




\subsection{Langauge Generation Tasks} \label{subsec:gpt}







\setlength{\tabcolsep}{2.5pt}
\renewcommand{\arraystretch}{0.75}
\begin{table}[thb]
\footnotesize
\caption{
Adaptive per-layer clipping matches the performance of flat clipping for full fine-tuning under the same number of training epochs for common privacy levels.
Results based on fine-tuning GPT-2 on E2E and DART.
Numbers in the column ``flat'' are reported by~\cite{li2022large}.
}
\centering
\begin{tabular}{l c cc cc}
\toprule
\multirow{2}[0]{*}{Metric} & \multirow{2}[0]{*}{DP Guarantee} & \multicolumn{2}{c}{E2E} & \multicolumn{2}{c}{DART} \\
\cmidrule(lr){3-4} \cmidrule(lr){5-6}
 & & {Adaptive per-layer} & {Flat} & {Adaptive per-layer} & {Flat} \\

\midrule
\multirow{3}[1]{*}{BLEU}

 & $\epsilon=3$ & 61.101 & 61.519     & 31.851 & 31.025  \\
 & $\epsilon=8$ & 63.416 & 63.189     & 34.166 & 35.057  \\
 & non-private    & -  & 69.463         & -  & 42.783  \\

\midrule
\multirow{3}[1]{*}{ROUGE-L}

 & $\epsilon=3$ & 65.120 & 65.670   & 51.519 & 52.063 \\
 & $\epsilon=8$ & 66.689 & 66.429   & 52.877 & 54.576 \\
 & non-private    & -  & 71.359         & -  & 56.717   \\

\bottomrule
\end{tabular}
\label{table:table2text_trimmed}
\end{table}

\paragraph{Table-To-Text Generation.}
We compare adaptive per-layer clipping against flat clipping for full fine-tuning with GPT-2 on the E2E~\citep{novikova2017e2e} and DART~\citep{nan2020dart} table-to-text generation tasks. 
Since~\cite{li2022large} performed extensive tuning on these tasks for flat clipping, we recall their results here.
For runs with adaptive per-layer clipping, we reused hyperparameter values tuned for SST-2, but re-tuned the target quantile with the E2E validation set and constrained the batch size and number of training epochs to be the same as that for flat clipping.
% ; see Appendix~\ref{app:lang_gen} for details.
Table~\ref{table:table2text_trimmed} confirms that DP learning with adaptive per-layer clipping performs comparably to flat clipping under a given epoch constraint.
\begin{wraptable}[14]{r}{6.3cm}
\footnotesize
\caption{Fine-tuning GPT-3 with DP LoRA achieves improved privacy-utility trade-off. Metrics are ROUGE scores.}
\vspace{-3mm}
\begin{tabular}{l c ccc}\\
\toprule
Model+Method+DP guarantee & R-1 & R-2 & R-L \\
\midrule
\textbf{Flat clipping} & & & \\
\hspace{1mm} GPT-2-xl LoRA ($\epsilon= 0.25$) & 8.0 & 2.7 & 6.8 \\
\hspace{1mm} GPT-2-xl LoRA ($\epsilon=1$)     & 30.0 & 11.0 & 25.3 \\
\hspace{1mm} GPT-2-xl LoRA ($\epsilon=4$)     & 35.4 & 14.3 & 29.9 \\
GPT-2-xl LoRA (non-private)     & 46.2 & 23.7 & 39.4 \\
\midrule
\textbf{Per-device clipping} & & & \\
\hspace{1mm} GPT-3 LoRA ($\epsilon= 0.25$) & 42.0 & 20.4 & 35.7 \\
\hspace{1mm} GPT-3 LoRA ($\epsilon= 1$)    & 48.0 & 25.4 & 41.3 \\
\hspace{1mm} GPT-3 LoRA ($\epsilon= 4$)    & 48.5 & 26.4 & 42.0 \\
GPT-3 LoRA (non-private) & 53.8 & 29.8 & 45.9 \\

\midrule
GPT-3 0-shot             & 27.4 & 9.0 & 20.9 \\
GPT-3 4-shot             & 42.1 & 19.6 & 34.2 \\

\bottomrule
\end{tabular}
\label{table:gpt3_trimmed}
\end{wraptable}

\paragraph{Dialog Summarization.}\label{exp:summarization}
We use the SAMSum dialog summarization task as a testbed for studying model scaling~\citep{gliwa2019samsum}.\footnote{We believe the chance of contamination occurring for this task to be small. See Appendix~\ref{app:per_device_clipping} for discussions.}
This task is more challenging than previously tested ones since its training set is small (less than 15k examples) and inputs are long. 
We fine-tune both GPT-2-xl and the (original) 175 billion-parameter GPT-3 with LoRA~\citep{hu2021lora} with and without DP, and compare them against in-context learning with GPT-3~\citep{brown2020language}.
Table~\ref{table:gpt3_trimmed} shows that GPT-3 fine-tuned at $\epsilon=1$ outperforms non-privately fine-tuned GPT-2-xl and in-context learning with 4 demonstrations (the maximum that can be fitted within the context window of 2048 tokens). 
See Appendix~\ref{app:per_device_clipping} for more details.
